\newpage
\begin{theo}[Specialised Woodbury Inversion Identity\footnote{This is directly stolen, with a few added steps, from~\cite{Thrun_gauss}.}] \label{theorem:woodbury}
For any invertible quadratic matrices $R$ and $Q$ and any matrix $P$ with appropriate dimensions, the following holds true
\begin{align*}
(R + P Q P^{T} )^{-1} &= R^{-1} - R^{-1} P (Q^{-1} + P^{T} R^{-1} P)^{-1} P^{T} R^{-1}
\end{align*}
\noindent \textbf{Proof}: Define $\Psi = (Q^{-1} + P^{T} R^{-1} P )^{-1}$. It suffices to show that
\begin{align*}
(R^{-1} - R^{-1} P \Psi P^{T} R^{-1})(R + P Q P) &= I 
\end{align*}
This is shown through a series of transformations
\begin{align*}
&= R^{-1} R - R^{-1} P Q P^{T} - R^{-1} P \Psi P^{T} R^{-1} R + R^{-1} P \Psi P^{T} R^{-1} P Q P^{T}  \\ 
&= I + R^{-1} P Q P^{T} - R^{-1} P \Psi P^{T} - R^{-1} P \Psi P^{T} R^{-1} P Q P^{T}  \\
&= I + R^{-1} P \left[ Q P^{T} - \Psi P^{T} - \Psi P^{T} R^{-1} P Q P^{T} \right]  \\
&= I + R^{-1} P \left[ Q P^{T} - \Psi Q^{-1} Q P^{T} - \Psi P^{T} R^{-1} P Q P^{T} \right]  \\
&= I + R^{-1} P \left[ Q P^{T} - \Psi \left[ Q^{-1} + P^{T} R^{-1} P \right] Q P^{T} \right]  \\
&= I + R^{-1} P \left[ Q P^{T} - \Psi \Psi^{-1} Q P^{T} \right] \\
&= I + R^{-1} P \left[ I - I \right] Q P^{T}  \\
&= I \nonumber
\end{align*}
\end{theo}